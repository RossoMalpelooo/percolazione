\section{Introduzione}

La teoria della percolazione nasce con l'obiettivo di ottenere un modello matematico 
per descrivere il fenomeno fisico della percolazione. Questo fenomeno descrive lo scorrimento 
di un fluido all'interno di un materiale tipicamente poroso. È importante specificare 
che il significato del termine ``percolazione'' può riferirsi a diversi contesti, 
a seconda del campo in cui si sta operando. Alcuni esempi sono: 
\begin{itemize}
    \item un soluto che diffonde attraverso un solvente;
    \item elettroni che migrano attraverso un reticolo atomico;
    \item molecole che penetrano un solido poroso;
    \item una malattia che infetta una comunità.
\end{itemize}
La formalizzazione matematica del problema ha reso possibile la creazione 
di un \textit{modello}. Un punto chiave molto importante di un modello 
è il concetto di \textit{astrazione}, caratteristica che permette di
operare senza considerare alcune variabili dipendenti dal contesto \cite{broadbent}.