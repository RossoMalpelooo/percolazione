\section{Conclusioni}

In conclusione, il progetto ha permesso di analizzare 
il fenomeno della percolazione attraverso un approccio frequentista, 
utilizzando due diversi algoritmi: uno standard $A$ e l'algoritmo di Hoshen-Kopelman. 
L'analisi statistica sui risultati ottenuti ha fornito una visione chiara dei 
meccanismi di percolazione e delle differenze di efficienza tra i due 
metodi implementati. Un naturale sviluppo futuro potrebbe essere lo studio 
del fenomeno su reticoli di forme diverse o in dimensioni superiori, come 
il caso tridimensionale, per esplorare eventuali variazioni nei comportamenti
osservati. Inoltre, un'interessante prospettiva di miglioramento sarebbe 
l'integrazione del codice Matlab con C++ tramite gli strumenti dedicati, 
permettendo una maggiore ottimizzazione delle strutture dati e 
un incremento dell’efficienza computazionale.